\documentclass[a4paper,10pt]{article}
\usepackage[utf8]{inputenc}
\usepackage{amsmath}
\usepackage{graphicx}
\usepackage{epstopdf}
\usepackage{textcomp}

\DeclareMathOperator*{\argmax}{arg\,max}
\DeclareMathOperator*{\argmin}{arg\,min}

%opening
\title{Tomografía óptica usando la ecuación independiente del tiempo 
de transferencia radiativa}
\author{A. Mendez}
\date{}

\begin{document}

\maketitle

%\begin{abstract}
%\end{abstract}

La tomografía óptica (OT), también denominada tomografía óptica difusa 
(DOT) o tomografía de migración de fotón (PMT), constituye una técnica
que permite reconstruir la distribución espacial de las propiedades
ópticas (por ejemplo, el coeficiente de absorción $\mu_a$ y el coeficiente 
de dispersión $\mu_s$) de ciertos medios dispersivos a partir de la 
medición de intensidades transmitidas y/o reflejadas en la superficie de
dicho medio. El rango de longitudes de onda de la luz utilizada para 
esta técnica está ubicado en el infrarojo cercano, aproximadamente
650nm$<\lambda<$900nm. 

La OT es implementada en diversas áreas científicas, desde la oceonografía 
hasta las ciencias atmosféricas, la astronomía y la física de neutrones. 
Sin embargo, los avances más recientes se deben a la implementación de la
OT en la óptica biomédica. Este campo está enfocado en el 
uso de la luz visible y cercana al infrarojo para el diagnóstico y 
tratamiento de tejidos biológicos. Existen ejemplos de monitoreo óptico 
de oxigenación de sangre, detección de hemorragias cerebrales, mapeo 
funcional de la actividad cerebral y diagnóstico de la enfermedad de 
Alzheimer, artritis reumoidal, o cáncer. La implementación de la OT se 
basa en el hecho de que ciertos procesos de las enfermedades y la mayoría 
de los cambios fisiológicos afectan las propiedades ópticas de los tejidos
biológicos. Las propiedades ópticas de interés son los coeficientes de 
absorción $\mu_a$, de dispersión $\mu_s$, y el factor de anisotropía~$g$, 
o una combinación de ellos. Las diferencias en estas propiedades ópticas 
entre tejidos sanos y patológicos proveen un contraste para esta 
tecnología de mapeo.

El transporte de fotones en medios dispersivos puede describirse a 
través de la ecuación independiente del tiempo de transferencia 
radiativa, dada por
\begin{equation}
 \omega\nabla\Psi(\mathbf{r},\omega)+(\mu_a+\mu_s)\Psi(\mathbf{r},\omega)
 =S(\mathbf{r},\omega) + 
 \mu_s\int_0^{2\pi}p(\omega,\omega')\Psi(\mathbf{r},\omega')\,d\omega'\,.
\label{eq:transf-rad}
\end{equation}
La cantidad fundamental en la teoría de transporte radiativo es la
radiancia $\Psi(\mathbf{r},\omega)$ en la posición espacial 
$\mathbf{r}$, sobre un unidad de ángulo sólido $\omega$, con unidades 
W cm$^{-2}$ sr$^{-1}$. La integral de la radiación sobre todos los 
ángulos $\omega$ en el punto $\mathbf{r}$ define la fluencia (densidad de 
energía) $\Phi(\mathbf{r})$:
\begin{equation}
 \Phi(\mathbf{r})=\int_{2\pi}\Psi(\mathbf{r},\omega)\,d\omega\,.
\end{equation}
Otras cantidades incluidas en la ecuación de transporte son los 
coeficientes de dispersión $\mu_s$ y de absorción $\mu_a$, ambos dados 
en unidades de cm$^{-1}$, y la función de fase de dispersión 
$p(\omega,\omega')$. En general, la función implementada es la función de
 dispersión de Henyey-Greennstein
\begin{equation}
 p(\cos\theta)=\frac{1-g^2}{2(1+g^2-2g\cos\theta)^{3/2}}\,
\end{equation}
donde $\theta$ es el ángulo entre las dos direcciones $\omega$ y 
$\omega$'. El factor $g$ se denomina factor de anisotropía y es utilizado 
para caracterizar la distribución angular de dispersión.

En general, la mayoría de los algoritmos implementados se basan en la validez 
de la aproximación de difusión de la ecuación de transferencia de radiación.
Sin embargo, esta aproximación ha mostrado tener límites experimentales y 
teóricos a la hora de describir la propagación de luz en tejidos biológicos.
Existen diversos métodos que permiten la reconstrucción de imagenes a partir 
de la tomografía óptica. Las técnicas más utilizadas se basan 
en la reconstrucción iterativa de imagenes basado en modelos (MOBIIR).
Estas técnicas implementan modelos {\it forward}, que proveen predicciones
en las lecturas del detector basándose en la estimación de la distribución
espacial de las propiedades ópticas del medio. Las lecturas predichas por 
el detector son comparadas con data experimental usando una función objetivo 
definida de manera apropiada. La verdadera distribución de las propiedades 
ópticas se determina actualizando iterativamente las predicciones y 
realizando nuevos cálculos hacia adelante con estas propiedades ópticas 
actualizadas hasta que los datos predichos coinciden con las lecturas del 
detector. Finalmente, la distribución de las propiedades ópticas se muestra 
como una imagen. 

\begin{thebibliography}{9}

\bibitem{Hielscher:98} 
Hielscher AH, Alcouffe RE, Barbour RL,
Phys Med Biol 43, 1285--302 (1998)

\textit{Klose:99}. 
Klose AD, Hielscher AH,
Med Phys 26(8), 1698--707 (1999)
 
\bibitem{Klose:02} 
Klose AD, Netz U, Beuthan J, Hielscher AH,
J Quant Spectrosc Radiat Transf 72, 691--713 (2002)

\end{thebibliography}



\end{document}
